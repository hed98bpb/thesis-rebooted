\section{Complexity of SPM map}

\begin{mydef}[Star-shaped polygon]\cite{PreparataS85}
	\label{star-shaped}
	A simple polygon $P$ is star-shaped if there exists a point $z$ not external
	to $P$ such that for all points $p$ of $P$ the line segment $\overline{zp}$
	lies entirely within $P$. The locus of the points $z$ having the above
	property is the \emph{kernel} of $P$.
\end{mydef}

\begin{Lemma}[Lemma 3.2]
	\label{lemma:3.2}
	The shortest path map $SPM(s)$ has $O(n)$ vertices, edges and faces. Each
	edge is a segment of a line or a hyperbola
\end{Lemma}
\begin{proof}
	Note that each face $SPM(s)$ is star-shaped (see definition
	\ref{star-shaped})
	with the unique predecessor vertex for the face, and the predecessor is in
	the kernel of the face.
	The idea behind this proof is to show that each obstacle vertex is a
	predecessor vertex for at most one face in $SPM(s)$.
	Consider a vertex $u$ that is the predecessor of a face $F$ and let
	$pred(u)$ be the set of predecessors of $u$, this is a set because there can
	be multiple predecessors. Observe that $d(s,u)=d(s,v)+|\overline{uv}|$ for
	any $v\in pred(u)$, since the distance $d(s,u)$ can always be rewritten as
	the distance to from $s$ to $u$'s predecessor, and a straight line from the
	predecessor to $u$ since your predecessor is always visible from a point.

	If a point $p$ is visible from a vertex $v \in pred(u)$ with $v$, $u$, $p$ 
    not being collinear, then $p$ cannot have $u$ as its predecessor. This is 
    due to the triangle inequality, where it is always shorter to take the direct 
    line instead going by another point.

	\nick{figure}

	Consider the subset of the free space that is visible from $u$
	but not visible from $v\in pred(u)$. Let $R(u,v)$ denote the component of
	this subset that is incident to $u$. Then $R(u,v)$ lies in an
	\nick{figure}
	angular angle around $u$ of less than 180$^\circ$. Define
	\begin{align}
		R(u) =  \bigcap_{v\in pred(u)} R(u,v)
	\end{align}

	Clearly $F \subseteq R(u)$, since the area $R(u)$ is the area that is
	incident to $u$, and not visible from any predecessor $v\in pred(u)$ 
	
	Then the claim is that there is at most one face of
	$SPM(s)$ in $R(u)$ with $u$ as its predecessor. 
	
	We do this by contradiction: 
	Suppose there were two faces
	$F_1$ and $F_2$ both having $u$ as their unique predecessor. The faces $F_1$
	and $F_2$ must have exactly one point in common, the vertex $u$. In the space
	between $F_1$ and $F_2$ there is a point $p$, there have to be a point here,
	otherwise they would be the same face. The point $p$ is arbitrarily close to $u$ with
	predecessor $z$ such that $z$ is distinct from both $u$ and $pred(u)$. In
	other words $d(s,u)+|\overline{up}|>d(s,z)+|\overline{zp}|$. However as $p$
	moves towards $u$ the difference in the distance shrinks and finally
	$d(s,u) = d(s,z)+|\overline{zu}|$. But then $z$ must be a predecessor of
	$u$. This means that $F_1$ and $F_2$ is part of the same face, contradicting
	the hypothesis. Thus a vertex $u$ is a predecessor of at most one face in
	the shortest path map.

	\nick{do the end from here to proof end}
	Finally to prove the linear upper bound on the size of the shortest path
	map, recall that the number of obstacle vertices is $n$ the remaining
	vertices border at least three faces of $SPM(s)$ (for this argument we count
	the obstacle polygons as faces of the shortest path map). Since the number
	of faces is $O(n)$, Eulers formula for planar graphs implies that the total
	number of vertices is also $O(n)$. This completes the proof
\end{proof}


\chapter{Introduction} Given a starting point $s$, an endpoint $t$ and a set of
polygons $P$, we want to find the shortest path from $s$ to $t$ without
traveling in the interior of any polygon in $P$. It is an old and well studied
problem, and historically there have been two approaches to the problem,
visibility graph and continuous Dijkstra. 

The visibility graph approach is to construct a graph of all legal paths
\nick{What is a legal path} and
then running a single source shortest path algorithm on that graph. The problem
with this approach is that the complexity of the graph can be $\Omega(n^2)$,
where $n$ is the number of vertices of the polygons. \nick{Sounds weird}

The continuous Dijkstra works by simulating a wave front
propagation\nick{define wavefront propagation} from the
start point $s$, where at simulation time $t$ the wavefront contains all the
points that is reached within distance $t$.  Every time a vertex edge is
reached a new wavefront starts emitting from there. In 1999 Hersberger and Suri
held a presentation in which they presented a $O(n\log n)$ time algorithm which
matched the lower bound of the problem, using the continuous Dijkstra method.

In 2017 they, together with Neeraj, looked at a generalization of that problem.
Given the same setting as before, you are now allowed to remove $k$ polygons.
Which polygons should you remove to minimize the distance from $s$ to $t$. They
presented an $O(k^2 n\log n)$ algorithm for this problem, which used a modified
version of the 1999 algorithm. Notice that if $k=0$ its the same problem as the
1999 one.

In this thesis we are going to look into implementations of the generalist
problem. We start by describing and implementing a naive algorithm for solving
the problem based on computing a visibility graph and then running the Dijkstra
single source shortest path algorithm. Afterwards we are going to explain the
implementation details and lastly we describe the algorithm from 2017 and
implement that.
\nick{Find ud af hvorfor visibility graphen er at gå fra vertex til
vertex,(trekants uligheden), og skriv det et sted her} \section{Problem
description} We a given two points in the plane $s,t\in\mathbb{R}^2$ and a list
of polygons $P=p_1,\dots,p_h$ where $p_i$ is a list of points in polygon $p_i$
starting an abitrary place and in clockwise order(note that sometimes we use
$P$ as a set f. ex. polygon $p_i\in P$). We say a legal path is a list of
points where two adjecent points are mutually visibible, i.e. you are able to
draw a line from one to the other without crossing the interioer of any polygon
in $P$.
We want to find a shortest legal path from $s$ to $t$.

\section{Naive algorithm}
Overview of naive
\section{Hersberger suri}
overview of hersberger suri
\section{Previous Work}
\
\begin{table}[H]
\begin{tabular}{ c c c c c c} 
	\hline
	Year & Paper & Run time & Space & Visibility graph & SPM \\
	\hline
	 & Naive & $(n^3)$ & $O(|E|)$ & x &\\
	1978 & Lee \cite{LEE78} & $O(n^2\log n)$ & ---\tablefootnote{We were not able
	to obtain the original ph.d. thesis} & x & \\
	1985 & Welzl \cite{DBLP:journals/ipl/Welzl85} & $O(n^2)$ & $O(n^2)$ & x & \\
	1991 & Ghosh et al. \cite{GhoshM91} & $O(E+n\log n)$\tablefootnote{Where $E$ is the
	number of edges in the visibility graph} &
	$O(E+n)$ & x &\\ %E is number of edges in the visibility graph
	1991 & Mitchell \cite{DBLP:journals/amai/Mitchell91} & $O(kn \log^2
	n)$\tablefootnote{Where $k$ is a number bounded by the number of different
	obstacles that touches any shortest path from $s$} & $O(n)$ & x\\
	1996 & Mitchell \cite{DBLP:journals/ijcga/Mitchell96} &
	$O(n^{3/2+\varepsilon})$\tablefootnote{For any $\varepsilon>0$ where the
	constant in the big-Oh notion depending on $\varepsilon$ } & $O(n)$ & & x\\
	1999 & Hershberger et al. \cite{HershbergerS99} & $O(n\log n)$ & $O(n\log n)$ & & x\\
	\hline
\end{tabular}
\caption{Shortest Paths in the Plane with Polygon obstacles algorithms}
\end{table}
\section{Overview of thesis}
\section{Preliminaries}

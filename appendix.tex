\chapter{Delaunay triangulation}\label{appendix:delaunaykruskal}

To understand what a Delaunay triangulation is, we start by understanding what a 
Voronoi digram is. The Voronoi digram is build around points in a plane, where 
the plane partition into a set of cells, where each cell has exactly one point 
in its interior. The special property for each of these cells is that each edge 
in there border is placed between two points, in such a way that the distance 
from the two points to any point on the edge is the same. The Voronoi digram can 
be computed in $O(n \log n)$ time \cite{CompGeo}. 

Delaunay Triangulation can then be understood as the dual graph of the Voronoi 
diagram. That is, we can build a graph, where the vertex in each cell of the 
Voronoi diagram gets a edge to another vertex if the vertex is a neighboring 
cell. This gives us a triangulation of the all the points in the plane, where
no edge overlaps.
\chapter{Survival guide}

\begin{figure}
\begin{center}
\begin{tabular}{| c | l |}
	\hline
	$n$ & number of vertices \\
	\hline
	$s$ & Starting point \\
	\hline
	$t$ & end point \\
	\hline
	$k$ & number of allowed violations \\
	\hline
	$P$ & the set of polygons \\
	\hline
\end{tabular}
\end{center}
\end{figure}

\chapter{Tables}
$k=5$, $n=0,..,42$
\begin{figure}
	\begin{tabular}{| c | l |}
		\hline
		n & time (microseconds)	\\
		\hline
		0 & 0\\
		\hline
		1 & 0\\
		\hline
		2 & 0\\
		\hline
		3 & 1\\
		\hline
		4 & 7\\
		\hline
		5 & 19\\
		\hline
		6 & 48\\
		\hline
		7 & 106\\
		\hline
		8 & 230\\
		\hline
		9 & 443\\
		\hline
		10 & 817\\
		\hline
		11 & 1431\\
		\hline
		12 & 2380\\
		\hline
		13 & 3816\\
		\hline
		14 & 5914\\
		\hline
		15 & 8928\\
		\hline
		16 & 13042\\
		\hline
		17 & 18820\\
		\hline
		18 & 26353\\
		\hline
		19 & 36337\\
		\hline
		20 & 49176\\
		\hline
		21 & 65528\\
		\hline
		22 & 86477\\
		\hline
		23 & 112619\\
		\hline
		24 & 144957\\
		\hline
		25 & 185109\\
		\hline
		26 & 233848\\
		\hline
		27 & 292542\\
		\hline
		28 & 363661\\
		\hline
	29 & 449106\\
	\hline
30 & 548799\\
\hline
31 & 667232\\
\hline
32 & 805750\\
\hline
33 & 970500\\
\hline
34 & 1158224\\
\hline
35 & 1376560\\
\hline
36 & 1628900\\
\hline
37 & 1918616\\
\hline
38 & 2251452\\
\hline
39 & 2630222\\
\hline
40 & 3058033\\
\hline
41 & 3546503\\
\hline
42 & 4094086\\
\hline
\end{tabular}
\end{figure}

\begin{figure}
	\begin{tabular}{| c | l |}
		\hline
		k & time (microseconds)\\
		\hline
		0 & 184927\\
		\hline
		1 & 185173\\
		\hline
		2 & 185082\\
		\hline
		3 & 185218\\
		\hline
		4 & 184888\\
		\hline
		5 & 184901\\
		\hline
		6 & 185203\\
		\hline
		7 & 184971\\
		\hline
		8 & 185074\\
		\hline
		9 & 185320\\
		\hline
		10 & 185211\\
		\hline
		11 & 185254\\
		\hline
		12 & 185479\\
		\hline
		13 & 185317\\
		\hline
		14 & 185491\\
		\hline
		15 & 185355\\
		\hline
		16 & 185504\\
		\hline
		17 & 185928\\
		\hline
		18 & 185985\\
		\hline
		19 & 185669\\
		\hline
		20 & 185842\\
		\hline
		21 & 185879\\
		\hline
		22 & 185980\\
		\hline
		23 & 186128\\
		\hline
		24 & 186250\\
		\hline
		25 & 186189\\
		\hline
	\end{tabular}
\end{figure}
\begin{figure}
	\begin{tabular}{| c | l |}
		\hline
		k \& n & time (microseconds)\\
		\hline
		0 & 0\\
		\hline
		1 & 0\\
		\hline
		2 & 0\\
		\hline
		3 & 1\\
		\hline
		4 & 8\\
		\hline
		5 & 22\\
		\hline
		6 & 48\\
		\hline
		7 & 109\\
		\hline
		8 & 237\\
		\hline
		9 & 452\\
		\hline
		10 & 833\\
		\hline
		11 & 1459\\
		\hline
		12 & 2434\\
		\hline
		13 & 3897\\
		\hline
		14 & 6081\\
		\hline
		15 & 9064\\
		\hline
		16 & 13331\\
		\hline
		17 & 19042\\
		\hline
		18 & 26792\\
		\hline
		19 & 36930\\
		\hline
		20 & 50024\\
		\hline
		21 & 66827\\
		\hline
		22 & 87957\\
		\hline
		23 & 114600\\
		\hline
		24 & 147841\\
		\hline
		25 & 187993\\
		\hline
	\end{tabular}
\end{figure}

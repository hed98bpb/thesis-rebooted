\chapter{Delaunay triangulation}\label{appendix:delaunaykruskal}

To understand what a Delaunay triangulation is, we start by understanding what a 
Voronoi digram is. The Voronoi digram is build around points in a plane, where 
the plane partition into a set of cells, where each cell has exactly one point 
in its interior. The special property for each of these cells is that each edge 
in there border is placed between two points, in such a way that the distance 
from the two points to any point on the edge is the same. The Voronoi digram can 
be computed in $O(n \log n)$ time \cite{CompGeo}. 

Delaunay Triangulation can then be understood as the dual graph of the Voronoi 
diagram. That is, we can build a graph, where the vertex in each cell of the 
Voronoi diagram gets a edge to another vertex if the vertex is a neighboring 
cell. This gives us a triangulation of the all the points in the plane, where
no edge overlaps.
\chapter{Survival guide}

\begin{figure}
\begin{center}
    \addtolength{\leftskip} {-2.5cm} % increase (absolute) value if needed
    \addtolength{\rightskip}{-2.5cm}
\begin{tabular}{| c | l |}
	\hline
	$n$ & number of vertices \\
	\hline
	$k$ & number of violation \\
	\hline
	$l_i,l$ & edge segments \\
	\hline
	$e,e_i,f,f_i$ & edges \\
	\hline
	$i,j$ & counting variables \\
	\hline
	$p,p_i,q,q_i,a,b,r$ & points \\
	\hline
	$s$ & stars point, source \\
	\hline
	$t$ & end point \\
	\hline
	$V$ & set of vertices, in the context of $G(E,V)$, else all vertices in the plane including $s$\\
	\hline
	$E$ & set of edges \\
	\hline
	$G$ & Graph \\
	\hline
	$p_{i.x}, q_{i.x}$ & $x$ coordinate of points \\
	\hline
	$p_{i.y}, q_{i.y}$ & $4$ coordinate of points\\
	\hline
	$v_i$ & vectors or vertices \\
	\hline
	$A,B,C$ & Figures, like triangles and square, or corners of triangles \\
	\hline
	$L_i$ & lines \\
	\hline
	$v_{\pi}$ & predecessor \\
	\hline
	$s_d, v_d$ & upper bound of weight of shortest paths \\
	\hline
	$w(\cdot, \cdot), w(\cdot)$ & weight function with one or two inputs \\
	\hline
	$Q$ & min priority queue \\
	\hline
	$\mathcal{S}$ & straight line subdivision \\
	\hline
\end{tabular}
\end{center}
\end{figure}


\begin{figure}
\begin{center}
    \addtolength{\leftskip} {-2.5cm} % increase (absolute) value if needed
    \addtolength{\rightskip}{-2.5cm}
\begin{tabular}{| c | l |}
    \hline
	$\alpha$ & well covering parameter \\
	\hline
	$C(e)$ & set of cells with $e$ in its interior \\
	$\mathcal{C}_i(e_i)$ & set of cells of $\mathcal{S}_i$ whose union $\mathcal{U}_{\alpha}(e_\alpha)$ the the well covering region $e_{\alpha}$ \\ 
	\hline
	$\mathcal{U}(e)$ & $\{c \mid c \in \mathcal{C}(e)\}$ \\
	\hline
	$c$ & cell \\
    \hline
	$\Delta$ & side length of figure \\
	\hline
	$x,y$ & $= j \cdot 2^i$ coordinate of $i$-quad \\
	\hline
	$b$ & $i$-box \\
	\hline
	$\mathcal{Q}(i)$ & $i$-quads at stage $i$ \\
	\hline
	$S, S', S_j(i)$ & equivalence class of $\mathcal{Q}(i)$ \\
	\hline
	$\equiv_i$ & transitive equivalence relation of overlap \\
    \hline
	$input(e)$ & edges who's wavefront is used for computing distance to $e$ from $\mathcal{U}(e)$'s boundary \\
	\hline
	$output(e)$ & $input(e) \cup \{f \mid e \in input(f)\}$ \\
	\hline
	$\partial$ & boundary \\
	\hline
	$V_S$ & points in the cores $S$ quads \\
	\hline
	$\overline{pq}$ & line segment between point $p$ and $q$ \\
	\hline
	$covertime(e)$ & time where $e$ is fully covered \\
	\hline
	$w,w'$ & wavelet \\
	\hline
	$O, O_i$ & obstacle \\
	\hline
	$\mathcal{O}$ & the set of obstacles in the plane \\
	\hline
	$g,g'$ & generators \\
	\hline
	$\pi(p)$ & set of shortest path from $s$ to $p$ \\
	\hline
	$\pi_k(p)$ & the shortest path from $s$ to $p$ \\
	\hline
	$SPM$ & shortest path map \\
	\hline
	$SPM_k$ & shortest $k$-path map\\
	\hline
	$T, T_i$ & minimum spanning tree or subset of edges \\
	\hline
	$T'$ & joined tree of $T_1$ and $T_2$ \\
	\hline
	$T_x$ & one of $\{T_1, T_2\}$ \\
	\hline
	$\mathcal{N}$ & set of minimum spanning trees \\
	\hline
	$G(i)$ & graph with set of vertices $V$ with edges weight of less than $6 \cdot 2^i$ \\
	\hline
	$|\cdot|$ & length of edges of size of set \\
	\hline
	$R_1$ & $\bigcup_{q \in S'} \{\text{the core of } growth(q)\}$ \\
	\hline
	$R_2$ & $\bigcup_{q \in S'} \{\text{the region covered by } q\}$ \\
	\hline
	$M$ & matching in graph \\
	\hline
	$q_u, q_v$ & $i$-quad q containing vertex $v$ or $u$ in its core \\
	\hline
	$MSF(\cdot)$ & minimum spanning forest at stage $i$ \\
	\hline
	$d(\cdot, \cdot)$ & distance function \\
	\hline
	$\Tilde{d}(s,e)$ & $\min(d(s,a),d(s,b))$ \\
	\hline
	$W(e)$ & approximate wavefront passing through edge $e$ \\
	\hline
	$W(f,e)$ & $\{w(f) \mid f \in input(e)\}$ \\
	\hline
	$W(f',e)$ & topologically different than $W(f,e)$ \\
	\hline
	$S(e)$ & S-face which segment intersects $e$ \\
	\hline
	S-face & piece of active region
	\hline
\end{tabular}
\end{center}
\end{figure}
\chapter{Tables}
$k=5$, $n=0,..,42$
\begin{figure}
	\begin{tabular}{| c | l |}
		\hline
		n & time (microseconds)	\\
		\hline
		0 & 0\\
		\hline
		1 & 0\\
		\hline
		2 & 0\\
		\hline
		3 & 1\\
		\hline
		4 & 7\\
		\hline
		5 & 19\\
		\hline
		6 & 48\\
		\hline
		7 & 106\\
		\hline
		8 & 230\\
		\hline
		9 & 443\\
		\hline
		10 & 817\\
		\hline
		11 & 1431\\
		\hline
		12 & 2380\\
		\hline
		13 & 3816\\
		\hline
		14 & 5914\\
		\hline
		15 & 8928\\
		\hline
		16 & 13042\\
		\hline
		17 & 18820\\
		\hline
		18 & 26353\\
		\hline
		19 & 36337\\
		\hline
		20 & 49176\\
		\hline
		21 & 65528\\
		\hline
		22 & 86477\\
		\hline
		23 & 112619\\
		\hline
		24 & 144957\\
		\hline
		25 & 185109\\
		\hline
		26 & 233848\\
		\hline
		27 & 292542\\
		\hline
		28 & 363661\\
		\hline
	29 & 449106\\
	\hline
30 & 548799\\
\hline
31 & 667232\\
\hline
32 & 805750\\
\hline
33 & 970500\\
\hline
34 & 1158224\\
\hline
35 & 1376560\\
\hline
36 & 1628900\\
\hline
37 & 1918616\\
\hline
38 & 2251452\\
\hline
39 & 2630222\\
\hline
40 & 3058033\\
\hline
41 & 3546503\\
\hline
42 & 4094086\\
\hline
\end{tabular}
\end{figure}

\begin{figure}
	\begin{tabular}{| c | l |}
		\hline
		k & time (microseconds)\\
		\hline
		0 & 184927\\
		\hline
		1 & 185173\\
		\hline
		2 & 185082\\
		\hline
		3 & 185218\\
		\hline
		4 & 184888\\
		\hline
		5 & 184901\\
		\hline
		6 & 185203\\
		\hline
		7 & 184971\\
		\hline
		8 & 185074\\
		\hline
		9 & 185320\\
		\hline
		10 & 185211\\
		\hline
		11 & 185254\\
		\hline
		12 & 185479\\
		\hline
		13 & 185317\\
		\hline
		14 & 185491\\
		\hline
		15 & 185355\\
		\hline
		16 & 185504\\
		\hline
		17 & 185928\\
		\hline
		18 & 185985\\
		\hline
		19 & 185669\\
		\hline
		20 & 185842\\
		\hline
		21 & 185879\\
		\hline
		22 & 185980\\
		\hline
		23 & 186128\\
		\hline
		24 & 186250\\
		\hline
		25 & 186189\\
		\hline
	\end{tabular}
\end{figure}
\begin{figure}
	\begin{tabular}{| c | l |}
		\hline
		k \& n & time (microseconds)\\
		\hline
		0 & 0\\
		\hline
		1 & 0\\
		\hline
		2 & 0\\
		\hline
		3 & 1\\
		\hline
		4 & 8\\
		\hline
		5 & 22\\
		\hline
		6 & 48\\
		\hline
		7 & 109\\
		\hline
		8 & 237\\
		\hline
		9 & 452\\
		\hline
		10 & 833\\
		\hline
		11 & 1459\\
		\hline
		12 & 2434\\
		\hline
		13 & 3897\\
		\hline
		14 & 6081\\
		\hline
		15 & 9064\\
		\hline
		16 & 13331\\
		\hline
		17 & 19042\\
		\hline
		18 & 26792\\
		\hline
		19 & 36930\\
		\hline
		20 & 50024\\
		\hline
		21 & 66827\\
		\hline
		22 & 87957\\
		\hline
		23 & 114600\\
		\hline
		24 & 147841\\
		\hline
		25 & 187993\\
		\hline
	\end{tabular}
\end{figure}

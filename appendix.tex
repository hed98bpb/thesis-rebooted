\chapter{Delaunay triangulation}\label{appendix:delaunaykruskal}

To understand what a Delaunay triangulation is, we start by understanding what a 
Voronoi digram is. The Voronoi digram is build around points in a plane, where 
the plane partition into a set of cells, where each cell has exactly one point 
in its interior. The special property for each of these cells is that each edge 
in there border is placed between two points, in such a way that the distance 
from the two points to any point on the edge is the same. The Voronoi digram can 
be computed in $O(n \log n)$ time \cite{CompGeo}. 

Delaunay Triangulation can then be understood as the dual graph of the Voronoi 
diagram. That is, we can build a graph, where the vertex in each cell of the 
Voronoi diagram gets a edge to another vertex if the vertex is a neighboring 
cell. This gives us a triangulation of the all the points in the plane, where
no edge overlaps.
\chapter{Survival guide}

\begin{figure}
\begin{center}
    \addtolength{\leftskip} {-2.5cm} % increase (absolute) value if needed
    \addtolength{\rightskip}{-2.5cm}
\begin{tabular}{| c | l |}
	\hline
	$n$ & number of vertices \\
	\hline
	$k$ & number of violation \\
	\hline
	$l_i,l$ & edge segments \\
	\hline
	$e,e_i,f,f_i$ & edges \\
	\hline
	$i,j$ & counting variables \\
	\hline
	$p,p_i,q,q_i,a,b,r$ & points \\
	\hline
	$s$ & stars point, source \\
	\hline
	$t$ & end point \\
	\hline
	$V$ & set of vertices, in the context of $G(E,V)$, else all vertices in the plane including $s$\\
	\hline
	$E$ & set of edges \\
	\hline
	$G$ & Graph \\
	\hline
	$p_{i.x}, q_{i.x}$ & $x$ coordinate of points \\
	\hline
	$p_{i.y}, q_{i.y}$ & $4$ coordinate of points\\
	\hline
	$v_i$ & vectors or vertices \\
	\hline
	$A,B,C$ & Figures, like triangles and square, or corners of triangles \\
	\hline
	$L_i$ & lines \\
	\hline
	$v_{\pi}$ & predecessor \\
	\hline
	$s_d, v_d$ & upper bound of weight of shortest paths \\
	\hline
	$w(\cdot, \cdot), w(\cdot)$ & weight function with one or two inputs \\
	\hline
	$Q$ & min priority queue \\
	\hline
	$\mathcal{S}$ & straight line subdivision \\
	\hline
\end{tabular}
\end{center}
\end{figure}


\begin{figure}
\begin{center}
    \addtolength{\leftskip} {-2.5cm} % increase (absolute) value if needed
    \addtolength{\rightskip}{-2.5cm}
\begin{tabular}{| c | l |}
    \hline
	$\alpha$ & well covering parameter \\
	\hline
	$C(e)$ & set of cells with $e$ in its interior \\
	$\mathcal{C}_i(e_i)$ & set of cells of $\mathcal{S}_i$ whose union $\mathcal{U}_{\alpha}(e_\alpha)$ the the well covering region $e_{\alpha}$ \\ 
	\hline
	$\mathcal{U}(e)$ & $\{c \mid c \in \mathcal{C}(e)\}$ \\
	\hline
	$c$ & cell \\
    \hline
	$\Delta$ & side length of figure \\
	\hline
	$x,y$ & $= j \cdot 2^i$ coordinate of $i$-quad \\
	\hline
	$b$ & $i$-box \\
	\hline
	$\mathcal{Q}(i)$ & $i$-quads at stage $i$ \\
	\hline
	$S, S', S_j(i)$ & equivalence class of $\mathcal{Q}(i)$ \\
	\hline
	$\equiv_i$ & transitive equivalence relation of overlap \\
    \hline
	$input(e)$ & edges who's wavefront is used for computing distance to $e$ from $\mathcal{U}(e)$'s boundary \\
	\hline
	$output(e)$ & $input(e) \cup \{f \mid e \in input(f)\}$ \\
	\hline
	$\partial$ & boundary \\
	\hline
	$V_S$ & points in the cores $S$ quads \\
	\hline
	$\overline{pq}$ & line segment between point $p$ and $q$ \\
	\hline
	$covertime(e)$ & time where $e$ is fully covered \\
	\hline
	$w,w'$ & wavelet \\
	\hline
	$O, O_i$ & obstacle \\
	\hline
	$\mathcal{O}$ & the set of obstacles in the plane \\
	\hline
	$g,g'$ & generators \\
	\hline
	$\pi(p)$ & set of shortest path from $s$ to $p$ \\
	\hline
	$\pi_k(p)$ & the shortest path from $s$ to $p$ \\
	\hline
	$SPM$ & shortest path map \\
	\hline
	$SPM_k$ & shortest $k$-path map\\
	\hline
	$T, T_i$ & minimum spanning tree or subset of edges \\
	\hline
	$T'$ & joined tree of $T_1$ and $T_2$ \\
	\hline
	$T_x$ & one of $\{T_1, T_2\}$ \\
	\hline
	$\mathcal{N}$ & set of minimum spanning trees \\
	\hline
	$G(i)$ & graph with set of vertices $V$ with edges weight of less than $6 \cdot 2^i$ \\
	\hline
	$|\cdot|$ & length of edges of size of set \\
	\hline
	$R_1$ & $\bigcup_{q \in S'} \{\text{the core of } growth(q)\}$ \\
	\hline
	$R_2$ & $\bigcup_{q \in S'} \{\text{the region covered by } q\}$ \\
	\hline
	$M$ & matching in graph \\
	\hline
	$q_u, q_v$ & $i$-quad q containing vertex $v$ or $u$ in its core \\
	\hline
	$MSF(\cdot)$ & minimum spanning forest at stage $i$ \\
	\hline
	$d(\cdot, \cdot)$ & distance function \\
	\hline
	$\Tilde{d}(s,e)$ & $\min(d(s,a),d(s,b))$ \\
	\hline
	$W(e)$ & approximate wavefront passing through edge $e$ \\
	\hline
	$W(f,e)$ & $\{w(f) \mid f \in input(e)\}$ \\
	\hline
	$W(f',e)$ & topologically different than $W(f,e)$ \\
	\hline
	$S(e)$ & S-face which segment intersects $e$ \\
	\hline
	S-face & piece of active region
	\hline
\end{tabular}
\end{center}
\end{figure}
\chapter{Tables}
\begin{table}[H]
\begin{tabular}{ |c|c| } 
 \hline
	n & time (in miliseconds) \\
	\hline
	2 & 0 \\
	\hline
	6 & 0 \\
	\hline
	18 & 0 \\
	\hline
	38 & 1 \\
	\hline
	66 & 9 \\
	\hline
	102 & 24 \\
	\hline
	146 & 50 \\
	\hline
	198 & 112 \\
	\hline
	258 & 235 \\
	\hline
	326 & 466 \\
	\hline
	402 & 848 \\
	\hline
	486 & 1481 \\
	\hline
	578 & 2462 \\
	\hline
	678 & 3939 \\
	\hline
	786 & 6083 \\
	\hline
	902 & 9271 \\
	\hline
	1026 & 13357 \\
	\hline
	1158 & 19278 \\
	\hline
	1298 & 27520 \\
	\hline
	1446 & 37018 \\
	\hline
	1602 & 50129 \\
	\hline
	1766 & 66930 \\
	\hline
	1938 & 88378 \\
	\hline
	2118 & 114971 \\
	\hline
	2306 & 148055 \\
	\hline
	2502 & 188859 \\
	\hline
	2706 & 238534 \\
	\hline
	2918 & 298675 \\
	\hline
	3138 & 371052 \\
	\hline
	3366 & 457771 \\
	\hline
	3602 & 559747 \\
	\hline
	3846 & 680632 \\
	\hline
	4098 & 822848 \\
	\hline
	4358 & 989762 \\
	\hline
	4626 & 1182116 \\
	\hline
	4902 & 1404667 \\
	\hline
	5186 & 1663503 \\
	\hline
	5478 & 1958169 \\
	\hline
	5778 & 2296425 \\
	\hline
	6086 & 2682430 \\
	\hline
	6402 & 3120744 \\
	\hline
	6726 & 3616376 \\
	\hline
\end{tabular}
\caption{Data from graph 1}
\end{table}

\begin{table}[H]
\begin{tabular}{ |c|c|c|c| } 
 \hline
	$n$ & crossing & visibility & Dijkstra\\
	\hline
	2 & 0 & 0 & 0 \\
	\hline
	6 & 0 & 0 & 0 \\
	\hline
	18 & 0 & 0 & 0 \\
	\hline
	38 & 1 & 0 & 0 \\
	\hline
	66 & 8 & 0 & 1 \\
	\hline
	102 & 22 & 0 & 2 \\
	\hline
	146 & 45 & 1 & 4 \\
	\hline
	198 & 104 & 2 & 6 \\
	\hline
	258 & 222 & 3 & 10 \\
	\hline
	326 & 447 & 4 & 15 \\
	\hline
	402 & 821 & 6 & 21 \\
	\hline
	486 & 1444 & 9 & 28 \\
	\hline
	578 & 2415 & 11 & 36 \\
	\hline
	678 & 3878 & 15 & 46 \\
	\hline
	786 & 6006 & 18 & 59 \\
	\hline
	902 & 9178 & 22 & 71 \\
	\hline
	1026 & 13244 & 28 & 85 \\
	\hline
	1158 & 19144 & 33 & 101 \\
	\hline
	1298 & 27359 & 40 & 121 \\
	\hline
	1446 & 36821 & 47 & 150 \\
	\hline
	1602 & 49895 & 56 & 178 \\
	\hline
	1766 & 66658 & 65 & 207 \\
	\hline
	1938 & 88071 & 75 & 232 \\
	\hline
	2118 & 114616 & 87 & 268 \\
	\hline
	2306 & 147650 & 100 & 305 \\
	\hline
	2502 & 188395 & 115 & 349 \\
	\hline
	2706 & 237995 & 131 & 408 \\
	\hline
	2918 & 298074 & 148 & 453 \\
	\hline
	3138 & 370383 & 167 & 502 \\
	\hline
	3366 & 457025 & 190 & 556 \\
	\hline
3602 & 558914 & 213 & 620 \\
\hline
3846 & 679704 & 238 & 690 \\
\hline
4098 & 821822 & 264 & 762 \\
\hline
4358 & 988634 & 298 & 830 \\
\hline
4626 & 1180886 & 328 & 902 \\
\hline
4902 & 1403311 & 363 & 993 \\
\hline
5186 & 1662007 & 400 & 1096 \\
\hline
5478 & 1956547 & 440 & 1182 \\
\hline
5778 & 2294661 & 481 & 1283 \\
\hline
6086 & 2680522 & 530 & 1378 \\
\hline
6402 & 3118674 & 574 & 1496 \\
\hline
6726 & 3614138 & 628 & 1610 \\
\hline
\end{tabular}
\caption{Data from graph 2}
\end{table}

\begin{table}[H]
\begin{tabular}{ |c|c| } 
 \hline
$n$ & time in miliseconds dividet by $n^3$ \\
\hline
2 & 0 \\
\hline
6 & 0 \\
\hline
18 & 0 \\
\hline
38 & 1.82242309374544E-05 \\
\hline
66 & 3.13047833708991E-05 \\
\hline
102 & 2.26157360291291E-05 \\
\hline
146 & 1.60661359272217E-05 \\
\hline
198 & 1.44285421297971E-05 \\
\hline
258 & 1.36838638480003E-05 \\
\hline
326 & 1.34503354733029E-05 \\
\hline
402 & 1.3053221060855E-05 \\
\hline
486 & 1.29016795495295E-05 \\
\hline
578 & 1.27498340864401E-05 \\
\hline
678 & 1.26385397648696E-05 \\
\hline
786 & 1.25270894447943E-05 \\
\hline
902 & 1.26330137388433E-05 \\
\hline
1026 & 1.2367070702209E-05 \\
\hline
1158 & 1.24147019560424E-05 \\
\hline
1298 & 1.25841634982224E-05 \\
\hline
1446 & 1.2243570102851E-05 \\
\hline
1602 & 1.21927454179994E-05 \\
\hline
1766 & 1.2152027041557E-05 \\
\hline
1938 & 1.21417937429069E-05 \\
\hline
2118 & 1.21006985351175E-05 \\
\hline
2306 & 1.20738331437455E-05 \\
\hline
2502 & 1.20580136097767E-05 \\
\hline
2706 & 1.20383485707373E-05 \\
\hline
2918 & 1.20210667777948E-05 \\
\hline
3138 & 1.20081459851104E-05 \\
\hline
3366 & 1.20034459584254E-05 \\
\hline
3602 & 1.19773474781993E-05 \\
\hline
3846 & 1.19642236814143E-05 \\
\hline
4098 & 1.19564913967957E-05 \\
\hline
4358 & 1.19582904652676E-05 \\
\hline
4626 & 1.19410690659842E-05 \\
\hline
4902 & 1.19248646628465E-05 \\
\hline
5186 & 1.19268580687987E-05 \\
\hline
5478 & 1.19119836093996E-05 \\
\hline
5778 & 1.19047328401354E-05 \\
\hline
6086 & 1.1899605218836E-05 \\
\hline
6402 & 1.18935399250374E-05 \\
\hline
6726 & 1.18851040850164E-05 \\
\hline
\end{tabular}
\caption{Data from graph 3}
\end{table}
\begin{table}[H]
\begin{tabular}{ |c|c| } 
	\hline
	$k$ & time in miliseconds \\
	\hline
	0 & 184927 \\
	\hline
	1 & 185173 \\
	\hline
	2 & 185082 \\
	\hline
	3 & 185218 \\
	\hline
	4 & 184888 \\
	\hline
	5 & 184901 \\
	\hline
	6 & 185203 \\
	\hline
	7 & 184971 \\
	\hline
	8 & 185074 \\
	\hline
	9 & 185320 \\
	\hline
	10 & 185211 \\
	\hline
	11 & 185254 \\
	\hline
	12 & 185479 \\
	\hline
	13 & 185317 \\
	\hline
	14 & 185491 \\
	\hline
	15 & 185355 \\
	\hline
	16 & 185504 \\
	\hline
	17 & 185928 \\
	\hline
	18 & 185985 \\
	\hline
	19 & 185669 \\
	\hline
	20 & 185842 \\
	\hline
	21 & 185879 \\
	\hline
	22 & 185980 \\
	\hline
	23 & 186128 \\
	\hline
	24 & 186250 \\
	\hline
	25 & 186189 \\
	\hline
\end{tabular}
\caption{Data from graph 4}
\end{table}

\chapter{Nick TODO!}
\begin{itemize}
	\item !DONE! skriv rå data ind i tabel i appendix
	\item skriv indledning
	\item lav historisk gennemgang
	\item !DONE! skriv noget til graferne
\end{itemize}

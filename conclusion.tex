In this thesis we have studied the problem of finding the shortest path in a plane with $k$ 
polygonal obstacle violations. First we showed a naive way for computing this problem, by 
building a visibility graph and using Dijkstra to find the shortest path with a maximum of 
$k$ allowed obstacle violation. This could be done in $O(n^3 + k \cdot n^2 + \text{Dijkstra})$, 
where Dijkstra runs in time $O((V + E) \log V)$. We implemented our naive algorithm a found 
our implementation ran in the expected time. 

Next we present an optimal solution of solving the shortest path problem without obstacle violation, 
which is due to Hershberger and Suri\cite{HershbergerS99}, and presented an extension of this solution 
into one which could handle paths with obstacle violation, which is due to Hershberger, Kumar and Suri
\cite{HershbergerKS17}. 

We began our discussion of the Hershberger-Suri algorithm with the construction of a conforming 
subdivision which builds a grid on top of the plane which give us some strong properties for subdividing 
it into a shortest path math. Here we considered both theoretical results and implementation details. 
Next we presented the wavefront propagation which uses the conforming subdivision to construct 
the shortest path map, from which we can query the shortest path between a source point $s$ 
and an endpoint $t$ in time $O(\log n)$ \cite{DBLP:journals/siamcomp/Kirkpatrick83}. 

Next we presented the extension of the Hershberger-Suri algorithm, with the needed modification
to the Hershberger-Suri algorithm and the overall algorithm which solves the shortest path with
$k$ polygonal obstacle violations in $k^2 \cdot n \log n)$ time. We also gave an overview of the
theory behind the solution to give a better understanding of the main ideas on which the algorithm
is build upon.

At last we gave a reduction of finding a shortest path without obstacle violation to sorting numbers
and proved the optimally of the Hershberger-Suri algorithm.

\section{Future work}

Implementing the Hershberger Suri algorithm as described through chapters \ref{chapter:conformingsubdivision} and 
\ref{chapter:wavefrontpropagation} would seem quite natural as the next step in this process. The sheer complexity
of it seems to make this a non trivial task. 

\section*{Acknowledgement}

We want to thank Niels Bross and Kresten Maigaard Axelsen for being great office buddies and also being up for a nice chat and a cup of coffee. 

We also want to thank Andrease Malling Østergaard and Andreas Østergaard Nielsen for being great skribbl.io opponents. 

